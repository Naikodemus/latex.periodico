Este texto es el que se espera encontrar en una noticia. Se van cofigurando poco a poco los elementos necesarios. Títulos, subtítulos, columnas, imágenes sencillas. Incluso los elementos destacados.

Forma parte del trabajo de diseño/edición escoger el emplazamiento de los destaques, si los hay. La separación entre párrafos puede hace que no sea conveniente.

\destaque{Ni os lo imagináis.}

Hay un problema asociado a la extensión de las columnas. Al dimensionar siempre la logitud total de la caja de texto de forma variable, las dos columnas irán siempre a la par. Ambas llegarán, o casi, al final inferior de la caja. Estoy puede causar que a veces haya espacios de salto de párrafo extra entre los párrafos de la última columna. No hay diferencia, por lo tanto, entre \enquote{multicols} y \enquote{multicols*}. La manera de solventar este problema es mediante \enquote{vspace} o \enquote{phantom}.

\lipsum[1-6]

Y aquí acaba la noticia.