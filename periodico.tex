%!TEX TS-program = xelatex
%!TEX encoding = UTF-8 Unicode

% Construcción de una revista/periódico en LaTeX
% uso personal

% PREÁMBULO
\documentclass[12pt]{article}
\usepackage[
%	showframe,
	a4paper,
%		width=210mm,
%		height=297mm,
	left=10.5mm,
	right=10.5mm,
	top=19.5mm,
	bottom=22.5mm
	]{geometry}
\usepackage[spanish]{babel}
\usepackage{csquotes}


% LOGICA 
\usepackage{calc}
\usepackage{ifthen}


% TIPOGRAFÍAS Y MOTOR
\usepackage{fontspec}
	\usepackage{xltxtra}
	\usepackage{xunicode}
\defaultfontfeatures{
%	Ligatures=Historic,
%	Style=Historic,
	Kerning=On,
	Style=TitlingCaps,
	Mapping=tex-text % ligadura de los guiones
	}
\setmainfont{Junicode}

\usepackage{marvosym}
\usepackage{lettrine}


% IMÁGENES Y COLORES
\usepackage{graphicx}
	\DeclareGraphicsExtensions{.pdf,.png,.jpg,.eps}
	\graphicspath{{img/}}
\usepackage[usenames,dvipsnames]{xcolor}
	\definecolor{DarkGray}{gray}{0.35}
	\definecolor{AshGray}{rgb}{0.7, 0.75, 0.71}
	\definecolor{Caja}{gray}{0.80}
	\definecolor{Rojo}{rgb}{1.0, 0, 0}
	\definecolor{ForestGreen}{rgb}{0, 0.27, 0.13}
\usepackage{tikz}
\usepackage{picinpar}
\usepackage[font={color=DarkGray,footnotesize},
	skip=2.55mm,
	labelformat=empty,
	labelsep=newline]{caption}
\usepackage{wrapfig}


% MAQUETACIÓN
\usepackage{lipsum}

% ENCABEZADOS PÁGINAS
\usepackage{fancyhdr}
%!TEX root = ../index.tex
%!TEX encoding = UTF-8 Unicode

	\renewcommand{\headrulewidth}{0pt}
	\renewcommand{\footrulewidth}{0pt}

% EMPTY
\fancypagestyle{empty}{
	\fancyhf{}
}

% PLAIN
\fancypagestyle{plain}{
	\fancyhf{}
}

% PORTADA
\fancypagestyle{portada}{
	\fancyhf{}
	\fancyfoot[C]{
		{\color{AshGray}\rule{\textwidth}{0.5pt}}\\[-1mm]
		{\color{Black}\footnotesize [~\thepage~]}%
	}
}

% PAGINA
\fancypagestyle{pagina}{
	\fancyhf{}
	\fancyhead[L]{
		{\color{Black}\footnotesize\epoca -- \numero}\\[-3mm]
	}
	\fancyhead[C]{
		{\color{Black}\footnotesize \nombrepub}\\[-3mm]
		{\color{AshGray}\rule{\textwidth}{0.5pt}}%
	}
	\fancyhead[R]{
		{\color{Black}\footnotesize\fechapub}\\[-3mm]
	}
	\fancyfoot[C]{
		{\color{AshGray}\rule{\textwidth}{0.5pt}}\\[-1mm]
		{\color{Black}\footnotesize [~\thepage~]}%
	}
}


% MANEJO TEXTO
\usepackage[
	absolute,
%	showboxes
	]{textpos}
	\TPGrid[5.5mm,5.5mm]{6}{9}
	\TPMargin{5mm}
\usepackage{multicol}
\usepackage{ragged}

\parindent=1em
\newcommand{\indentplinea}{-15.5pt}

% ENTORNOS Y COMANDOS
%!TEX root = ../index.tex
%!TEX encoding = UTF-8 Unicode

% MEMO
% \newcommand{\comandonuevo}{prog}

% FIN NOTICIA
\newcommand{\fin}{\color{Black}\SquareSteel}

% NUEVA PÁGINA
\newcommand{\pasapag}{\null\clearpage}

% PORTADA
\newcommand{\frontal}{
	\thispagestyle{portada}
	\begin{textblock}{6}(0,-0.125)
		{\fontspec{Alfios}
		 \fontsize{42}{0}\selectfont
		 \color{Black}
		 \hfill\textbf{\nombrepub}\hfill
		}\\[1.5mm]
		{\small\color{Gray}
			\hfill{\MakeUppercase\lema}\hfill
		}\\[-1.5mm]
		{\color{AshGray}\rule{\textwidth}{0.5pt}}\\
		\hspace*{1mm}
		{\small\color{DarkGray}
			\makebox[.3\textwidth][l]{\epoca\ -- \numero}%
			\hfill
			\makebox[.3\textwidth][c]{\fechapub}%
			\hfill
			\makebox[.3\textwidth][r]{\indic}%
			\hspace{1mm}
		}\\[-3.0mm]
		{\color{AshGray}\rule{\textwidth}{0.5pt}}%
	\end{textblock}
	}


% NOTICIAS
\newcommand{\titulonoticia}[3]{%
	\ifthenelse{\equal{#1}{}}
		{}
		{{\raggedright\rmfamily\huge\textbf{#1}\par}}
	\ifthenelse{\equal{#2}{}}
		{\vspace{#3mm}}
		{{\raggedright\rmfamily\Large{\color{DarkGray} #2}\par\vspace{#3mm}}}%
}


% ENTRADA
\newcommand{\entrada}[2]
	{\ifthenelse{\equal{#2}{}}
		{\textbf{\uppercase{#1.}}}
		{\textbf{\uppercase{#1.} #2.}}
	}


% DESTACADOS
\newcommand{\destaque}[1]{%
	\vspace{-1.25\baselineskip}%
	\begin{center}%
	\hrulefill\\[0.25\baselineskip]
	{\fontspec{DejaVu Sans}\scriptsize\bfseries#1}\\[-0.25\baselineskip]
	\hrulefill
	\end{center}%
	\vspace{-0.75\baselineskip}
}

\newcommand{\caja}[2]{%
	\noindent\fcolorbox{White}{Caja}{%
		\parbox[b][#1em][c]{\linewidth}{\small\sffamily\textbf{#2}}
	}
}


% IMAGENES
\newcommand{\img}[5]{%
	\begin{figure}[H]
	\vspace{#4mm}
	\centering
	\noindent\includegraphics[width=#2\linewidth]{img/#1}
	\ifthenelse{\equal{#3}{}}{}{\caption{#3}}
	\vspace{#5mm}
	\end{figure}
}

% FILETES
\newcommand{\fileteV}[3]{%
	\begin{textblock}{0}(#1,#2)
	\color{AshGray}\line(0,-1){#3}
	\end{textblock}
}
\newcommand{\fileteH}[3]{%
	\begin{textblock}{0}(#1,#2)
	\color{AshGray}\line(1,0){#3}
	\end{textblock}
}


%!TEX root = ../index.tex
%!TEX encoding = UTF-8 Unicode

% MEMO
% \newenvironment{Nombre}[Num Args.][Arg. 1]{Al entrar}{Al salir}

% NOTICIAS
\newenvironment{noticiacol1}[6]
	{\begin{textblock}{#1}(#2,#3)%
	 \titulonoticia{#4}{#5}{#6}
	 \hspace{\indentplinea}%
	}
	{\fin\end{textblock}
	}

\newenvironment{noticiacolmulti}[8]
	{\begin{textblock}{#1}(#2,#3)%
	 \titulonoticia{#4}{#5}{#6}%
	 #8\begin{multicols*}{#7}
	 \hspace{\indentplinea}%
	}
	{\fin\end{multicols*}
	 \end{textblock}
	}

% CAJAS
\newenvironment{cajacol1}[6]
	{\begin{textblock}{#1}(#2,#3)%
	 \titulonoticia{#4}{#5}{#6}
	 \hspace{\indentplinea}%
	}
	{\end{textblock}
	}

\newenvironment{cajacolmulti}[8]
	{\begin{textblock}{#1}(#2,#3)%
	 \titulonoticia{#4}{#5}{#6}
	 #8\begin{multicols*}{#7}
	 \hspace{\indentplinea}%
	}
	{\end{multicols*}
	 \end{textblock}
	}


% PDF - ENLACES
\usepackage[hidelinks]{hyperref}


% DATOS DE LA PUBLICACIÓN * * * * * * * * * * * * * * * * * * * * * * *
\newcommand{\nombrepub}{COSAS QUE PASAN}
\newcommand{\lema}{y son raras}
\newcommand{\fechapub}{\today}
\newcommand{\epoca}{EÓN I}
\newcommand{\numero}{número 1}
\newcommand{\indic}{cc-by-sa}


% CUERPO * * * * * * * * * * * * * * * * * * * * * * * * * * * * * * * *
\begin{document}

% HACER

\frontal

% esta es la primera página/portada
%!TEX root = ../index.tex
%!TEX encoding = UTF-8 Unicode

\begin{noticiacolmulti}{4}{0}{1}
	{Noticia con imagen insertada}
	{En el texto}
	{-2}	% separación entre títulos y cuerpo de texto
	{2}		% cantidad de columna
	{}
	\noindent\entrada{Ciudad}{Autor}
	Vamos con una noticia que incluya imagen en el cuerpo de texto, pero que no sea de cabecera. Que para eso habrá que hacer otro entorno distinto y anterior al despliege de las múltiples columnas.

\img{dollar}{0.95}{}{2}{0}

Hay que cuidar el reparto del texto. Si es insuficiente, el espacio alrededor de la imagen puede quedar excesivo.

\lipsum[1]

Y con ésto acaba el texto.
\end{noticiacolmulti}

\begin{noticiacol1}{2}{4}{1}
	{Noticia breve}
	{Subtítulo}
	{2} % separación entre títulos y cuerpo de texto
	\small
	\entrada{Ciudad}{}
	Ejemplo de noticia a una sola columna. No muy extensa, no muy corta. Hay una dependencia importante sobre las extensiones de los textos.

\img{eye}{0.75}{Todo lo ve, todo.}{3}{0}

Se emplea para esta noticia el entorno \enquote{\textbf{noticiacol1}}, que permite especificar 6 variables ancho de la caja, posición horizontal y vertical de la caja de texto, título y subtítulo, y separación en milímetros entre los títulos y el cuerpo de texto.

\destaque{Esto es un texto destacado.}

También emplea un comando personalizado para la imagen, de cinco (5) variables: nombre de la imagen, ancho porcentual de la línea, pie de foto, y separación superior e inferior.

Para conseguir que la justificación funcione se ha disminuido el tamaño de la letra un paso.
\end{noticiacol1}

\begin{noticiacolmulti}{4}{0}{5.75}
	{Este es un buen título de la noticia}
	{Los protagonistas la liaron en cuanto pudieron}
	{-2}	% separación entre títulos y cuerpo de texto
	{2}		% cantidad de columnas
	{}
	\entrada{Ciudad}{Autor}
	\input{news/noticia01}
\end{noticiacolmulti}

\begin{cajacol1}{2}{4}{6.30}
	{}
	{}
	{0} % separación entre títulos y cuerpo de texto
	\img{aurum}{1}{V.I.T.R.I.O.L.}{3}{0}
\noindent
Este texto es el de la medallita de...
\end{cajacol1}

\fileteV{3.85}{1}{675}
\fileteH{0}{5.55}{325}
\fileteH{4}{6.20}{137.5}\pasapag

% este es el resto del periódico
\pagestyle{pagina}
%!TEX root = ../index.tex
%!TEX encoding = UTF-8 Unicode

\begin{cajacol1}{2}{0}{0.15}
	{}
	{}
	{0} % separación entre títulos y cuerpo de texto
	\caja{2}{Se descoloca el lado derecho por estar en latín.}
\end{cajacol1}

\begin{noticiacolmulti}{4}{2}{0.20}
	{Este es un buen título de la noticia}
	{Los protagonistas la liaron en cuanto pudieron}
	{-2}	% separación entre títulos y cuerpo de texto
	{2}		% cantidad de columnas
	{\img{003}{1}{En el laboratorio}{5}{-5}}	% a insertar antes de las columnas
	\entrada{Ciudad}{Autor}
	\lipsum[1-2]
	\destaque{El lío fue gordo, nadie se esperaba nada parecido.}
	\lipsum[2]
	Y así muere una noticia...
%	\input{news/noticia06}
\end{noticiacolmulti}

\begin{noticiacol1}{2}{0}{0.60}
	{¡Columna!}
	{}
	{2} % separación entre títulos y cuerpo de texto
	\noindent\small\lipsum[2-3]\lipsum[2]
	Y aquí acaba la cosa.
%	\input{news/noticia06}
\end{noticiacol1}

\fileteV{1.85}{0.140}{760}\pasapag
%!TEX root = ../index.tex
%!TEX encoding = UTF-8 Unicode

\begin{noticiacolmulti}{6}{0}{0.20}
	{Este es otro buen título de la noticia}
	{Los protagonistas la liaron bien liada}
	{-2} % separación entre títulos y cuerpo de texto
	{3} % cantidad de columnas
	{}
	\entrada{Ciudad}{Autor}
	Este texto es el que se espera encontrar en una noticia. Se van cofigurando poco a poco los elementos necesarios. Títulos, subtítulos, columnas, imágenes sencillas. Incluso los elementos destacados.

Forma parte del trabajo de diseño/edición escoger el emplazamiento de los destaques, si los hay. La separación entre párrafos puede hace que no sea conveniente.

\destaque{Ni os lo imagináis.}

Hay un problema asociado a la extensión de las columnas. Al dimensionar siempre la logitud total de la caja de texto de forma variable, las dos columnas irán siempre a la par. Ambas llegarán, o casi, al final inferior de la caja. Estoy puede causar que a veces haya espacios de salto de párrafo extra entre los párrafos de la última columna. No hay diferencia, por lo tanto, entre \enquote{multicols} y \enquote{multicols*}. La manera de solventar este problema es mediante \enquote{vspace} o \enquote{phantom}.

\lipsum[1-6]

Y aquí acaba la noticia.
\end{noticiacolmulti}

\begin{cajacol1}{5.9}{0}{8.175}
	{}
	{}
	{0}
	\caja{2.5}{Este es un texto corto sobre la noticia en una caja para indicar algo puntual sobre la noticia. Pero es una caja sin otro tipos de marcadores, así que puede actuar de faldón.}
\end{cajacol1}

\end{document}

% EOF